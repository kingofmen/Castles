\documentclass[12pt,ebook,oneside]{book}

\usepackage{graphicx} 
\usepackage{layouts} 
\usepackage{hyperref} 
\textheight=1.2\textheight

\begin{document}


\section{Hex display}

\subsection{Militia}

\subsection{Village}

\subsection{Cows}

Supplies available to a Hex are depicted as cows. The maximum number
of cows is given by the \texttt{maxCows} parameter in the
\texttt{cowPositions} object of \texttt{gfx/info.txt}. Their positions
are shown in the same object, relative to one corner of the pasture
field of the hex. The number of supplies represented by one cow is
$M_st/M_c$, where $M_S$ is the highest number of supplies in the game
at start, $t$ is the \texttt{cowTolerance} parameter of the
\texttt{cowPositions} object, and $M_c$ is the \texttt{maxCows}
parameter. 

\subsection{Buttons}

There are buttons for map modes, supply priority (field units,
garrisons in castles), recruit types (castles), and drill level
(hexes). Their size is given by \texttt{buttonSize} in
\texttt{CastleWindow.cpp}. Their position is set when they are
created, to hardcoded values. Their icons are set by
\texttt{setUItexts} in \texttt{StaticInitialiser}, which gets the
information from \texttt{gui.txt} - this also supplies the tooltips.

\section{Production}

Each hex has a Village containing the population and militia; a
Farmland containing its fields; a Forest which may produce wood; and a
Mine which may produce iron. 

Each turn, the Hex economy runs thus:
\begin{itemize}
\item The village produces labour from its population.
\item Producers calculate their raw needs.
\item Contracts are executed.
\item Owners decide what to deliver from their reserves,
and set their bids and offers for surpluses and needs.
\item Hold a market using the bids and offers.
\item Production.
\item Produced goods are delivered to owners.
\item Taxation.
\item Consumption, including investment.
\item Contract renegotiation.
\end{itemize}

\subsection{Labour production}

The village produces labour in proportion to its population, weighted
for age and sex. The production of each age bracket is set in
\texttt{common/popInfo.txt}, in the \texttt{production} field. Females
are weighted by the \texttt{femaleProduction} field - male production
is taken as 1. Militia drill subtracts from this amount by the drill
level, multiplied by the number of soldiers in each unit, multiplied
by the \texttt{militiaDrill} field of the unit type definition. 

\end{document}
