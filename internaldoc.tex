\documentclass[12pt,ebook,oneside]{book}

\usepackage{graphicx} 
\usepackage{layouts} 
\usepackage{hyperref} 
\textheight=1.2\textheight

\begin{document}


\section{Hex display}

\subsection{Militia}

\subsection{Village}

\subsection{Cows}

Supplies available to a Hex are depicted as cows. The maximum number
of cows is given by the \texttt{maxCows} parameter in the
\texttt{cowPositions} object of \texttt{gfx/info.txt}. Their positions
are shown in the same object, relative to one corner of the pasture
field of the hex. The number of supplies represented by one cow is
$M_st/M_c$, where $M_S$ is the highest number of supplies in the game
at start, $t$ is the \texttt{cowTolerance} parameter of the
\texttt{cowPositions} object, and $M_c$ is the \texttt{maxCows}
parameter. 

\subsection{Buttons}

There are buttons for map modes, supply priority (field units,
garrisons in castles), recruit types (castles), and drill level
(hexes). Their size is given by \texttt{buttonSize} in
\texttt{CastleWindow.cpp}. Their position is set when they are
created, to hardcoded values. Their icons are set by
\texttt{setUItexts} in \texttt{StaticInitialiser}, which gets the
information from \texttt{gui.txt} - this also supplies the tooltips.


\end{document}
