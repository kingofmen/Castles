\documentclass[12pt,ebook,oneside]{book}

\usepackage{graphicx} 
\usepackage{layouts} 
\usepackage{hyperref} 
\textheight=1.2\textheight

\begin{document}

\section{Hex display}

\subsection{Militia}

\subsection{Village}

\subsection{Cows}

Supplies available to a Hex are depicted as cows. The maximum number
of cows is given by the \texttt{maxCows} parameter in the
\texttt{cowPositions} object of \texttt{gfx/info.txt}. Their positions
are shown in the same object, relative to one corner of the pasture
field of the hex. The number of supplies represented by one cow is
$M_st/M_c$, where $M_S$ is the highest number of supplies in the game
at start, $t$ is the \texttt{cowTolerance} parameter of the
\texttt{cowPositions} object, and $M_c$ is the \texttt{maxCows}
parameter. 

\subsection{Buttons}

There are buttons for map modes, supply priority (field units,
garrisons in castles), recruit types (castles), and drill level
(hexes). Their size is given by \texttt{buttonSize} in
\texttt{CastleWindow.cpp}. Their position is set when they are
created, to hardcoded values. Their icons are set by
\texttt{setUItexts} in \texttt{StaticInitialiser}, which gets the
information from \texttt{gui.txt} - this also supplies the tooltips.

\section{Production}

Each hex has a Village containing the population and militia; a
Farmland containing its fields; a Forest which may produce wood; and a
Mine which may produce iron. 

Each turn, the Hex economy runs thus:
\begin{itemize}
\item The village produces labour from its population.
\item Producers calculate their raw needs.
\item Contracts are executed.
\item Owners decide what to deliver from their reserves,
and set their bids and offers for surpluses and needs.
\item Hold a market using the bids and offers.
\item Production.
\item Produced goods are delivered to owners.
\item Taxation.
\item Consumption, including investment.
\item Contract renegotiation.
\end{itemize}

\subsection{Labour production}

The village produces labour in proportion to its population, weighted
for age and sex. The production of each age bracket is set in
\texttt{common/popInfo.txt}, in the \texttt{production} field. Females
are weighted by the \texttt{femaleProduction} field - male production
is taken as 1. Militia drill subtracts from this amount by the drill
level, multiplied by the number of soldiers in each unit, multiplied
by the \texttt{militiaDrill} field of the unit type definition. 

\section{Substitution}

This section outlines how the Village decides how much to buy and
consume of goods that are partial substitutes.

\subsection{Defining substitutes}

A MaslowLevel may have more than one TradeGood, with relative weights,
like so:
\begin{verbatim}
level = {
  food = 0.19
  iron = 0.01
  mortality = 0.96
  max_work_fraction = 0.94
  name = "Barely fed"
}
\end{verbatim}

\subsection{Buying substitutes}

Let $x_i$ be the amount of good $i$, $y_i$ its price, and $w_i$ its
weight, normalised so their sum is 1. Then we require
\begin{eqnarray}
\label{eq:constraint}
\prod\limits_i x_i^{w_i} &=& C
\end{eqnarray}
where $C$ is the Village's consumption factor, and minimise the total
price paid
\begin{eqnarray*}
P = \sum\limits_i x_iy_i
\end{eqnarray*}
under this constraint. To find the amounts to bid for we use a
Lagrange multiplier:
\begin{eqnarray*}
\Lambda(\vec x, \vec y, \vec w) &=& \sum\limits_i x_iy_i - \lambda\left(\prod\limits_i x_i^{w_i} - C\right)\\
\frac{\partial\Lambda}{\partial x_i} &=& y_i - \lambda w_ix^{w_i-1}\prod\limits_{j\ne i} x_j^{w_j}
\end{eqnarray*}
and setting the derivative to equal zero gives
\begin{eqnarray*}
\lambda w_ix_i^{-1}x_i^{w_i}\prod\limits_{j\ne i} x_j^{w_j} &=& y_i
\lambda &=& \frac{x_iy_i}{w_i}\prod\limits_i x_i^{-w_i}.
\end{eqnarray*}
We see that when equating the right-hand sides
the product drops out, and (defining in passing an effective price
$y_i'=y_i/w_i$) we get
\begin{eqnarray}
\label{eq:equalprice}
x_iy_i' &=& x_jy_j'
\end{eqnarray}
for all $i, j$. Inserting this back into equation~\ref{eq:constraint},
we find
\begin{eqnarray}
x_1^{w_1}\prod\limits_{j\ne 1} \left(\frac{x_1y_1'}{y_j'}\right)^{w_j} &=& C\\
x_1 &=& C \prod\limits_{j\ne 1}\left(\frac{y_1'}{y_j'}\right)^{-w_j} \\
\label{eq:amount}
    &=& Cy_1'^{w_1-1} \prod\limits_{j\ne 1}\left(y_j')^{w_j}\\
    &=& Cy_1'^{-1} \prod\limits_{j}\left(y_j')^{w_j}\\
    &=& \frac{Cw_1}{y_1} \prod\limits_{j}\left(y_j')^{w_j}
\end{eqnarray}
which looks fearsome but is straightforward to calculate; and from
there we can get the other $x_i$ by inserting into
equation~\ref{eq:equalprice}.

A sanity check: Suppose two goods have equal weights 0.5, and equal
prices 1; and suppose $C$ is 1000. Then $x_1$ is just 1000 and so is
$x_2$, which is sensible - equal goods, equal amounts, and equal to
the consumption capacity. Now suppose the first good has weight
two-thirds, and the prices are still equal. Then $x_1$ becomes 1260
and $x_2$ is 630, so the higher-weighted good is used more and the
total is still very close to 2000. Conversely if $y_1=0.75$ and
$y_2=1.5$, giving the same effective prices, then $x_1$ becomes 1414
and $x_2$ 707. These all seem like sensible numbers.

% (* (* 1000 (pow 1.5 -0.5) (pow 3 0.5)) 1.5 (/ 1.0 3))

\subsection{Consuming substitutes}

In this case there is no price to guide us. The constraint equation is
the same, since it reflects that the total consumption should be some
given constant. But the market prices don't necessarily reflect the
real consumption costs. So the new problem is to satisfy equation~\ref{eq:constraint}
while minimising the total fractional consumption $\sum\limits_i
x_i/X_i$ where $X_i$ is the reserve of good $i$. We see that this is
equivalent to just setting prices equal to inverse reserves, and we
can reuse equations~\ref{eq:equalprice} and \ref{eq:amount} with the
substitution $y_i' = \frac{1}{w_iX_i}$. By inspection we see that
consumption is directly proportional to weight and reserves, which
makes sense. To sanity-check the result, suppose we have equal
amounts, say 1000, of each good, equal weights, and consumption
capacity also 1000. Then $x_1$ is 1000 and so is $x_2$. Changing the
weights, as before, to two-thirds and one-third, we get 1260 and 630,
as before. Which is really bad because we assumed we had 1000 of
each. So we will have to\footnote{I tried doing the more general
  Karush-Kuhn-Tucker method, but I don't have the sufficient
  conditions for a solution to be optimal.} insert an additional clause: Don't use more
than you have; if forced to attempt it, use everything available and redo
the problem with one less good. However, on the assumption that we
have 2000 of each, we get $x_1$ of 1260 again and $x_2$ of 630, so
it's fine. If we have only 1000 of good 2, then we get 1587 and 397,
which is also ok.

%(let ((X1 2000)
%      (c1 1000)
%      (X2 1000) 
%      (w1 0.6666) 
%      (w2 (- 1 w1))
%      (x1 (* c1 (pow (/ 1.0 (* X1 w1)) (- w1 1)) (pow (/ 1.0 (* X2 w2)) w2)))
%      (x2 (* w2 X2 (/ 1.0 w1 X1))))
%(* x1 x2))



\end{document}
